% !TEX root = ../main.tex
\appendix
\chapter{Code Repository}

Ai fini di un adeguato meccanismo di versioning per la collaborazione tra gli studenti autori del progetto, si è scelto di utilizzare git come meccanismo di versioning.
Questo ha comportato un sistema affidabile per il tracciamento di versioni e modifiche, con la possibilità di tenere traccia, ed eventualmente facendo delle operazioni di revert, degli sviluppi incrementali dei vari progetti.
È stato reso disponibile il repository su github \footnote{il repository è disponibile su github all'indirizzo \url{https://github.com/andr3a87/NG_Cafe} }.	

\section{Il codice dei test}
Qui di seguito si trova il codice specifico per i vari test eseguiti sui domini implementati

\subsection{Cammini 10x10}

\begin{lstlisting}
occupata(pos(2,5)).
occupata(pos(3,5)).
occupata(pos(4,5)).
occupata(pos(5,5)).
occupata(pos(6,5)).
occupata(pos(7,5)).
occupata(pos(7,1)).
occupata(pos(7,2)).
occupata(pos(7,3)).
occupata(pos(7,4)).
occupata(pos(5,7)).
occupata(pos(6,7)).
occupata(pos(7,7)).
occupata(pos(8,7)).
occupata(pos(4,7)).
occupata(pos(4,8)).
occupata(pos(4,9)).
occupata(pos(4,10)).

iniziale(pos(4,2)).
finale(pos(7,9)).
\end{lstlisting}

\subsection{Cammini 20x20}

\begin{lstlisting}
occupata(pos(7,15)).
occupata(pos(8,15)).
occupata(pos(9,15)).
occupata(pos(10,15)).
occupata(pos(11,15)).
occupata(pos(12,15)).
occupata(pos(13,15)).

occupata(pos(13,6)).
occupata(pos(13,7)).
occupata(pos(13,8)).
occupata(pos(13,9)).
occupata(pos(13,10)).
occupata(pos(13,11)).
occupata(pos(13,12)).
occupata(pos(13,13)).
occupata(pos(13,14)).

occupata(pos(15,1)).
occupata(pos(15,2)).
occupata(pos(15,3)).
occupata(pos(15,4)).
occupata(pos(15,5)).
occupata(pos(15,6)).
occupata(pos(15,7)).
occupata(pos(15,8)).
occupata(pos(15,9)).


iniziale(pos(10,10)).

finale(pos(20,20)).
\end{lstlisting}

\subsection{Mondo dei blocchi del professor Martelli}

\begin{lstlisting}
block(a).
block(b).
block(c).
block(d).
block(e).

iniziale(S):-
  list_to_ord_set([on(a,b),on(b,c),ontable(c),clear(a),on(d,e), ontable(e),clear(d),handempty],S).

goal(G):- list_to_ord_set([on(a,b),on(b,c),on(c,d),ontable(d), ontable(e)],G).

finale(S):- goal(G), ord_subset(G,S).
\end{lstlisting}

\subsection{Mondo dei blocchi del professor Torasso}

\begin{lstlisting}
block(a).
block(b).
block(c).
block(d).
block(e).
block(f).
block(g).
block(h).


iniziale(S):-
        list_to_ord_set([clear(a), clear(c), clear(d), clear(e), clear(f), clear(g), clear(h), on(a,b),
        ontable(b), ontable(c), ontable(d), ontable(e), ontable(f), ontable(g), ontable(h), handempty],S).

goal(G):- list_to_ord_set([on(a,b),on(b,c),on(c,d),on(d,e),
        ontable(e)],G).
\end{lstlisting}

\subsection{La metropolitana di Londra}

Esempio 1:

\begin{lstlisting}
stazione('Baker Street',4.5,5.6).
stazione('Bank',12,4).
stazione('Bayswater',1,3.7).
stazione('Bond Street',5.4,4.1).
stazione('Covent Garden',8,4).
stazione('Earls Court',0,0).
stazione('Embankment',8.2,3).
stazione('Euston',7.1,6.6).
stazione('Gloucester Road',1.6,0.6).
stazione('Green Park',6,2.8).
stazione('Holborn',8.6,4.8).
stazione('Kings Cross',8.2,7.1).
stazione('Leicester Square',7.6,3.6).
stazione('London Bridge',0,0).
stazione('Notting Hill Gate',0,3.2).
stazione('Oxford Circus',6.2,4.3).
stazione('Paddington',2.4,4.2).
stazione('Piccadilly Circus',7,3.3).
stazione('South Kensington',2.6,0.5).
stazione('Tottenham Court Road',7.4,4.5).
stazione('Victoria',5.8,1).
stazione('Warren Street',6.5,6).
stazione('Waterloo',9.2,2.4).
stazione('Westminster',8,1.8).

iniziale([at('Bayswater'),ground]).

finale([at('Covent Garden'),ground]).
\end{lstlisting}

Esempio 2:

\begin{lstlisting}
iniziale([at('Holborn'),ground]).

finale([at('Waterloo'),ground]).
\end{lstlisting}
